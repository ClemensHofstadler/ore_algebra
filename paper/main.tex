\documentclass[a4paper,draft]{amsart}
\usepackage{amsmath}
\usepackage{amssymb}
\usepackage{bbm}
\usepackage{a4wide}

\parindent=0pt
\parskip\smallskipamount

\let\set\mathbbm
\def\<#1>{\langle#1\rangle}
\let\ideal\unlhd
\def\i{\mathrm{i}}
\def\e{\mathrm{e}}

\newtheorem{thm}{Theorem}
\newtheorem{prop}{Proposition}
\newtheorem{defn}{Definition}
\newtheorem{conj}{Conjecture}

\newcommand\todo[1][.]{\edef\tmpa{.}\edef\tmpb{#1}%
  \ifx\tmpa\tmpb
    \typeout{To Be on page \thepage}\fbox{\bf To Be}
  \else
    \typeout{To Be on page \thepage: #1}\fbox{{\bf To Be:} #1}
  \fi
}

\begin{document}

 \author[Manuel Kauers, Maximilian Jaroschek, Fredrik Johansson]
   {Manuel Kauers\,$^\ast$, Maximilian Jaroschek\,$^\ast$, Fredrik Johansson\,$^\ast$}
 \address{Manuel Kauers, Research Institute for Symbolic Computation (RISC), J. Kepler University Linz, Austria}
 \email{mkauers@risc.uni-linz.ac.at}
 \address{Maximilian Jaroschek, Research Institute for Symbolic Computation (RISC), J. Kepler University Linz, Austria}
 \email{mjarosch@risc.uni-linz.ac.at}
 \address{Fredrik Johansson, Research Institute for Symbolic Computation (RISC), J. Kepler University Linz, Austria}
 \email{fjohanss@risc.uni-linz.ac.at}
 \thanks{$^\ast$ Supported by the Austrian FWF grant Y464-N18.}

 \title{Ore Polynomials in Sage}

 \begin{abstract}
We present a Sage implementation of Ore algebras. The main features for the most
common instances include basic arithmetic and actions; gcrd and lclm; D-finite
closure properties; natural transformations between related algebras; guessing;
desingularization; solvers for polynomials, rational functions and (generalized)
power series. This paper is a tutorial on how to use the package.
 \end{abstract}

 \maketitle

%%% page limit: 20, goal: 15--20.

\section{Introduction}

motivation

similar packages: gfun, mallinger, q-gfun, future: mgfun, holonomicfunctions

overview over the package

short demo

outlook, pending tasks (multivariate!)

how to get it and how to install it. 

\section{Ore Algebras and Ore Polynomials}

\subsection{In Theory}

basic facts and definitions

reference to literature

\subsection{In Sage}

ground rings 

general sigma and delta

predefined shortcuts: D, T, S, F, J, Q, C

\section{General Methods}

arithmetic, lclm, gcrd, action, coercion and conversion, pretty printing, coefficient extraction,
closure properties, associate solutions. 

\section{Special Methods Standard Algebras}

special methods available for univ algebras with univ algebra as ground ring

\subsection{Differential Operators}

D and theta

\subsection{Recurrence Operators}

delta and S

\subsection{The $q$-Case}

J and Q

\section{Guessing}

\section{Further Examples}

Some longer session showing how to solve a meaningful problem by using several features of the code. Ideas?

\subsection{}

\subsection{}

 \bibliographystyle{plain}
 \bibliography{all}
 
\end{document}
